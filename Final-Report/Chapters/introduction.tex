\section{Introduction}
\label{sec:intro}
Sentiment analysis consists in performing sentence classification with the final goal to predict whether a sentence/review/comment expresses a positive or negative sentiment.\\ 
Sentiment analysis is a very important task in the field of Natural Language Processing (NLP) and has been studied for many years. It is a very challenging task because of the presence of negation, sarcasm, irony, and other linguistic phenomena that make the task very difficult.\\   
From a commercial point of view, sentiment analysis is a very important task because it can be used to automatically analyze the opinion of the customers about a product or a service. For the latter reason, many industries are interested in this specific applicatio to understand the degree of appreciation a specific product has.\\

As briefly mentioned before, in this work I'll be comparing two different models to perform Sentiment and polarity classification. 
\subsection{Baseline}
The baseline model consists in :
\begin{itemize}
    \item a \texttt{CountVectorizer}\cite{vectorizer}, which acts as an encoder, to extract the features from the text;
    \item a \texttt{Na\"{i}ve Bayes}\cite{naive} classifier to perform the final classification.
\end{itemize}
It is worth mentioning that both datasets were pre-processed applying double negative marking to consider double negations (double negations corresponds to a neutral effect).\\
More in detail, the step-by-step procedure followed to perform the classification is the following:
\begin{enumerate}
    \item Pre-processing: the text is pre-processed by collapsing the severale sentences in documents and the double negations are marked
    \item Feature extraction: the text belonging to the Subjectivity dataset is encoded using a \texttt{CountVectorizer} which counts the number of occurrences of each word in the text;
    \item Classification: the encoded text from the previous point is classified using a \texttt{Na\"{i}ve Bayes} classifier;
    \item Evaluation: the classification is evaluated using a 10-fold cross-validation precedure exloiting the \texttt{accuracy} metric to address the overall performances;
    \item Filtering : the trained model is used to filter the objective sentences from the Movie Review dataset;    
    \item Feature extraction: the text belonging to the Movie Review dataset is encoded using a \texttt{CountVectorizer} as point (2);
    \item Classification: the encoded text from the previous point is classified using a \texttt{Na\"{i}ve Bayes} classifier as point (3);
\end{enumerate}
\subsection{Proposed model}
The proposed model follows the same logical workflow as the baseline but with a completely different approach.\\
Very briefly, the proposed model consists in using :
\begin{itemize}
    \item a \texttt{Bert Tokenizer} \cite{tokenizer} to extract the features from the text and a \texttt{BertModel} \cite{model} to derive the embeddings;
    \item a shallow \texttt{BiLSTM} network used to filter objective movie reviews from the Movie Review dataset, accepting the output of the previous point as input;
    \item a \texttt{BertForSequenceClassification} to perform the final classification on the MovieReviews dataset as a whole.
\end{itemize}

