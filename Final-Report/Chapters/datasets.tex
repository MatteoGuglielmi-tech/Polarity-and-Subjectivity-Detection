% (approx. 200-500 words)

\vspace{-0.25cm}
\section{Data description and analysis}
\label{sec:data} 
\subsection{Movie reviews dataset}
\label{subsec:mr}
The movie reviews dataset used in this work is directly deployed by the \texttt{nltk.corpus} python package whose characteristics are the following :
\begin{itemize}
    \item number of words : \numprint{1583820};
    \item lexicon size : \numprint{39768};
    \item categories : 'neg', 'pos'.
\end{itemize}
Each movie review has a file id associated with it used for its identification containing all the relative information.
%The dataset presents itself in a peculiar structure. As mentioned to each review corresponds a file containing the relative information, all files are then added to a list, 
%each with an associated label.\\
This dataset is used just for evaluation, since the amount of reviews is too small to perform a proper fine-tuning operation of the BertForSequenceClassification (see 
\textbf{\Cref{subsec:custom}}).
%For this reason, the whole dataset has been used just to address the final polarity classifiaction accuracy.

\subsection{Subjectivity dataset}
\label{subsec:subj}
The subjectivity dataset is provided by the \texttt{nltk.corpus} python module as well.\\ 
The main characteristics of this dataset are :
\begin{itemize}
    \item vocabulary size : \numprint{240576};
    \item lexicon size : \numprint{23906};
    \item classes : 'obj', 'subj';
    \item total number of sentences : \numprint{10000}.
\end{itemize}
In this specific case, there are just two unique file ids corresponding to the documents containing the objective and subjective sentences. 
%The datastructure used in the code consists of the union of those two files.

\subsection{IMDB dataset}
\label{subsec:imdb}
The IMDB Dataset has \numprint{50000} movie reviews for natural language processing or text analytics ~\cite{kaggle}. This dataset can be used for binary sentiment 
classfication and it provides \numprint{25000} highly polar samples for training and \numprint{25000} samples for testing. More information reguarding this dataset can 
be found at ~\cite{refe}.
