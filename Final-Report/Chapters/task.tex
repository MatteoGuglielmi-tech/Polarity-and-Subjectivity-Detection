% (approx. 200 words)

\section{Task formalization}
Sentiment analysis is the use of natural language processing and text analysis to identify, extract and quantify affective states and subjective information.\\
As previously mentioned, sentiment analysis is deployed in different fields to give voice to the customers and for the companies to receive a feedbacks through the analysis of forums or reviews.\\
With the raise of deep learning, it is possible to deal with very difficult scenarios, e.g. when the opinions are not explicitly expressed.\\
In the specific case of this assignment, the goal is to build a machine learning/deep learning model capable of performing subjectivity and polarity classification comparing the achieved results to a baseline model.\\
More precisely, the aforementioned model needs to perform:
\begin{itemize}
    \item sentiment classification : the automated process of identifying opinions in text and labeling them as positive, negative, or neutral, based on the emotions expressed within them. In this specific case, the Subjectivity dataset presentes only the positive and negative tag reducing the problem to a binary classification task;
    \item polarity classification which consists in labeling a text as subjective or objective.
\end{itemize}


